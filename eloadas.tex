\input{preambulum.tex}

\newcommand{\git}{\textsc{git}}

\title{Bevezetés a \git használatába}
\author[Szolnoki Lénárd]{Szolnoki Lénárd}
\institute[WJSz]{Programozás szeminárium \\ Wigner Jenő Szakkollégium \\ BME TTK}


\begin{document}
\usebackgroundtemplate{%
\tikz[overlay,remember picture] \node[opacity=0.1, at=(current page.220)] {\includegraphics[height=\paperheight]{TTK_sz}};
}

\begin{frame}
 \titlepage
\end{frame}

\begin{frame}[noframenumbering]
 \frametitle{Tartalom}
 \tableofcontents
\end{frame}

\section{Bevezetés}
	\subsection{Motiváció}

\section{GIT}
	\subsection{Koncepció, a GIT feladata}
	\subsection{Absztrakciók, objektumok}
	\subsection{Parancsok}
	\subsection{Távoli elérés, Github.org}

\section{Gyakorlás, szituációk}
	\subsection{merge konfliktus}
	\subsection{Verziófa "metszése" (rebase)}
	\subsection{Pillanatnyi munka ideiglenes mentése (stash)}
	\subsection{Utolsó commit szerkesztése}
	\subsection{HEAD/branch mozgatása (reset)}

\section{Egyéb használati esetek}


\end{document}
