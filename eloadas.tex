\input{preambulum.tex}

\newcommand{\git}{\textsc{git} }

\title{Bevezetés a \git használatába}
\author[Szolnoki Lénárd]{Szolnoki Lénárd}
\institute[WJSz]{Programozás szeminárium \\ Wigner Jenő Szakkollégium \\ BME TTK}


\begin{document}
\usebackgroundtemplate{%
\tikz[overlay,remember picture] \node[opacity=0.1, at=(current page.220)] {\includegraphics[height=\paperheight]{TTK_sz}};
}

\begin{frame}
 \titlepage
\end{frame}

\begin{frame}[noframenumbering]
 \frametitle{Tartalom}
 \tableofcontents
\end{frame}

\section{Bevezetés}
	\subsection{Motiváció}
	\begin{frame}
	  Sokan találkozhattunk a problémával, hogyan tároljuk egy-egy munkánk egy adott időpillanatbeli verzióját.
	  \begin{itemize}
	    \item{content.bak}
	    \item{content.v1}
	    \item{\dots}
	  \end{itemize}
	  
	  Problémák:
	  \begin{itemize}
	    \item{Több fájl konzisztens verziókövetése}
	    \item{Verziók közötti változások nyomonkövetése}
	    \item{Kollaboráció}
	  \end{itemize}
	\end{frame}

\section{\git}
	\subsection{Koncepció, a \git feladata}
	\begin{frame}
	  A \git számunkra fontos feladatai:
	  \begin{itemize}
	    \item{Egy könyvtárstruktúrában tárolt forrásfájlokban tárolt adatok tárolása}
	    \item{Jól átlátható verziók közti struktúra}
	    \item{Többen tudjanak párhuzamosan dolgozni egy adott forráson}
	  \end{itemize}
	  Egyéb hasznos tulajdonságok:
	  \begin{itemize}
	    \item{Hatékony adattárolás}
	    \item{Beépített konzisztencia vizsgálat, biztonságos verziókövetés}
	  \end{itemize}
	\end{frame}
	\subsection{Absztrakciók, objektumok}
	\begin{frame}
	  Általános objektumok:
	  \begin{itemize}
	    \item{tree: egy egész forrásfa képe}
	    \item{commit: tree + egyéb metaadatok}
	      \begin{itemize}
		\item{Szülö azonosító(k) (ezek is commit objektumok)}
		\item{athor (név, e-mail)}
		\item{committer (név, e-mail)}
		\item{Dátum}
		\item{commit message}
	      \end{itemize}
	    \item{branch: egy commitre mutató névvel ellátott cimke}
	    \item{tag: egy commitre mutató névvel ellátott cimke (Később látjuk a különbséget)}
	  \end{itemize}
	  Egyedi objektumok:
	  \begin{itemize}
	    \item{HEAD: Az utolsó commitre mutató cimke/aktuális branch-re, a következő szülőjelölt}
	    \item{index: commitre váró tree}
	    \item{working directory: munkaterület (homokozó)}
	  \end{itemize}
	\end{frame}
	\subsection{Parancsok}
	\subsection{Távoli elérés, Github.org}

\section{Gyakorlás, szituációk}
	\subsection{merge konfliktus}
	\subsection{Verziófa "metszése" (rebase)}
	\subsection{Pillanatnyi munka ideiglenes mentése (stash)}
	\subsection{Utolsó commit szerkesztése}
	\subsection{HEAD/branch mozgatása (reset)}

\section{Egyéb használati esetek}


\end{document}
